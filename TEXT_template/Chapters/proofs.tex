\chapter{Proofs}\label{app:proofs}
In this appendix we provide the proofs that were omitted in Chapter \ref{chap:main}. Some auxiliary lemmas are also introduced and proved.

\begin{lemma}\label{aux:1}
For any initial state distribution $\mu$ and any pair of stationary Gaussian policies $\pi_{\vtheta} \sim \mathcal{N}(\vtheta^T\boldsymbol{\phi}(s),\sigma^2)$ and $\pi_{\vtheta'} \sim \mathcal{N}(\vtheta'^T\boldsymbol{\phi}(s),\sigma^2)$, so that $\vtheta'=\vtheta+\Lambda\gradJ{\vtheta}$, and for any state $s$ and action $a$:
\[
\pi_{\vtheta'}(a|s) - \pi_{\vtheta}(a|s) \geq \nabla_{\vtheta}\pi_{\vtheta}(a|s)^T\Lambda\gradJ{\vtheta} 
	- \frac{M_{\phi}^2\norm[1]{\Lambda\gradJ{\vtheta}}^2}{\sqrt{2\pi}\sigma^3}.
\]
\end{lemma}

\begin{proof}
By exploiting Taylor's expansion
\begin{align*}
\pi_{\vtheta'}(a|s) = & \pi_{\vtheta + \Lambda\gradJ{\vtheta}}(a|s) = \\
	& \pi_{\vtheta}(a|s) + \nabla_{\vtheta}\pi_{\vtheta}(a|s)^T\Delta\vtheta+R_1(\Delta\vtheta),
\end{align*}
where $R_1(\Delta\vtheta)$ is the remainder of the series, which can be bounded as follows by exploiting Lemma 4.1 from \cite{NIPS2013_5186} and Assumption 3.1:
\begin{align*}
R_1(\Delta\vtheta) = & \sum\limits_{i=1}^m\sum\limits_{j=1}^m
\frac{\partial^2\pi_{\vtheta}(a|s)}{\partial\theta_i\partial\theta_j} \bigg|_{\vtheta+c\Delta\vtheta} \frac{\Delta\theta_i\Delta\theta_j}{1+I(i=j)} & \text{for some $c \in (0,1)$} \\
	\geq & - \sum\limits_{i=1}^m\sum\limits_{j=1}^m \frac{|\phi_i(s)\phi_j(s)|}{\sqrt{2\pi}\sigma^3}
	\frac{\Delta\theta_i\Delta\theta_j}{1+I(i=j)} \\
	= & -\frac{1}{\sqrt{2\pi}\sigma^3}\sum\limits_{i=1}^m\sum\limits_{j=1}^m
	\frac{\alpha_i|\phi_i(s)|\gradJ{\theta_i}\alpha_j|\phi_j(s)|\gradJ{\theta_j}}{1+I(i=j)} \\
	= & - \frac{(|\gradJ{\vtheta}|^T\Lambda|\vphi(s)|)^2}{\sqrt{2\pi}\sigma^3} \\
	\geq & - \frac{M_{\phi}^2\norm[1]{\Lambda\gradJ{\vtheta}}^2}{\sqrt{2\pi}\sigma^3}.
\end{align*}
It is now enough to apply this bound to Taylor's expansion.
\end{proof}

\begin{lemma}\label{aux:2}
For any initial state distribution $\mu$ and any pair of stationary Gaussian policies $\pi_{\vtheta} \sim \mathcal{N}(\vtheta^T\boldsymbol{\phi}(s),\sigma^2)$ and $\pi_{\vtheta'} \sim \mathcal{N}(\vtheta'^T\boldsymbol{\phi}(s),\sigma^2)$, so that $\vtheta'=\vtheta+\Lambda\gradJ{\vtheta}$:

\[ \norm[\infty]{\pi_{\vtheta'}-\pi_{\vtheta}}^2 \leq \
	\frac{M_{\phi}^2\norm[1]{\Lambda\gradJ{\vtheta}}^2}{\sigma^2}.
\]
\end{lemma}

\begin{proof}
By exploiting Pinsker's inequality \cite{Pinsker1960}
\begin{align*}
\norm[\infty]{\pi_{\vtheta'}-\pi_{\vtheta}}^2 = & \sup\limits_s \norm[\infty]{\pi_{\vtheta'}-\pi_{\vtheta}}^2 \\
	\geq & \sup\limits_s 2H(\pi_{\vtheta'}||\pi_{\vtheta}) \\
	= & \sup\limits_s \frac{1}{\sigma^2} \sum\limits_i 
	(\gradJ{\theta_i}\alpha_i\phi_i(s))^2 \\
	\geq & \frac{M_{\phi}^2\norm[1]{\Lambda\gradJ{\vtheta}}^2}{\sigma^2},
\end{align*}
where $H(P||Q)$ is the Kullback-Liebler divergence.
\end{proof}


\firstlemma*
\begin{proof}
We plug the results of Lemmas \ref{aux:1} and \ref{aux:2} into Lemma 3.1 from \cite{NIPS2013_5186}:
\begin{align*}
J_{\mu}(\vtheta') - J_{\mu}(\vtheta) \geq  
	& \frac{1}{1-\gamma} \int_{\mathcal{S}}d_{\mu}^{\pi_{\vtheta}}(s)
	\int_{\mathcal{A}} (\pi_{\vtheta'}(a|s) - \pi_{\vtheta}(a|s))Q^{\pi_{\vtheta}}(s,a)\mathrm{d}a\mathrm{d}s \\
	& -\frac{\gamma}{2(1-\gamma)^2} \norm[\infty]{\pi_{\vtheta'}-\pi_{\vtheta}}^2 \norm[\infty]{Q^{\pi_{\vtheta}}} \\
	\geq & 
		\frac{1}{1-\gamma} \int_{\mathcal{S}}d_{\mu}^{\pi_{\vtheta}}(s)
		\int_{\mathcal{A}}
		\nabla_{\vtheta}\pi_{\vtheta}(a|s)^T\Lambda\gradJ{\vtheta}
		Q^{\pi_{\vtheta}}(s,a)\mathrm{d}a\mathrm{d}s && (1) \\
		& - \frac{M_{\phi}^2\norm[1]{\Lambda\gradJ{\vtheta}}^2}
		{(1-\gamma)\sqrt{2\pi}\sigma^3}
		\int_{\mathcal{S}}d_{\mu}^{\pi_{\vtheta}}(s)
		\int_{\mathcal{A}} 
		Q^{\pi_{\vtheta}}(s,a)\mathrm{d}a\mathrm{d}s && (2) \\ 
		&  -\frac{\gamma M_{\phi}^2\norm[1]{\Lambda\gradJ{\vtheta}}^2}
		{2(1-\gamma)^2\sigma^2}\norm[\infty]{Q^{\pi_{\vtheta}}}.	&& (3)
\end{align*}
Term $(1)$ can be simplified by using the Policy Gradient Theorem \cite{NIPS1999_1713}:
\begin{align*}
\frac{1}{1-\gamma} &  \int_{\mathcal{S}}d_{\mu}^{\pi_{\vtheta}}(s)
		\int_{\mathcal{A}}
		\nabla_{\vtheta}\pi(a|s,\vtheta)^T\Lambda\gradJ{\vtheta}
		Q^{\pi_{\vtheta}}(s,a)\mathrm{d}a\mathrm{d}s \\
		= & \frac{\gradJ{\vtheta}^T\Lambda}{1-\gamma}
		\int_{\mathcal{S}}d_{\mu}^{\pi_{\vtheta}}(s)
		\int_{\mathcal{A}}
		\nabla_{\vtheta}\pi(a|s,\vtheta)Q^{\pi_{\vtheta}}(s,a)\mathrm{d}a\mathrm{d}s \\
		= & \gradJ{\vtheta}^T\Lambda\gradJ{\vtheta}.
\end{align*}
The proof now follows simply by rearranging terms.
\end{proof}


% \paragraph{Theorem \ref{theo:1}}
\firsttheorem*
\begin{proof}
For every state $s \in \mathcal{S}$ and every action $a \in \mathcal{A}$, the Q-function belongs to $\left[-\frac{R}{1-\gamma},\frac{R}{1-\gamma}\right]$. As a consequence, $\int_{\mathcal{A}}Q^{\pi_{\vtheta}}(s,a)\mathrm{d}a \leq \frac{|\mathcal{A}|R}{1-\gamma}$ and $\norm[\infty]{Q^{\pi_{\vtheta}}} \leq \frac{R}{1-\gamma}$. The proof follows from applying these bounds to the expression of Lemma \ref{lemma:1}.
\end{proof}

% \paragraph{Theorem \ref{theo:3}}
\thirdtheorem*
\begin{proof}
The proof immediately follows from Theorem 3.3 and the definitions of $\gradDown{\vtheta}$ and $\gradUp{\vtheta}$ provided in Section \ref{sec:approx}. Note that the saturation to $0$ in $\gradDown{\vtheta}$ is necessary since $\gradJ{\vtheta}$ is taken with absolute value in the negative term of the original bound.
\end{proof}

%\paragraph{Theorem \ref{theo:5}}
\fourththeorem*
\begin{proof}
We first optimize the cost function $\Upsilon_{\delta}$ w.r.t $\Lambda$. Since $\Upsilon_{\delta}$ is just the bound from Theorem \ref{theo:3} divided by $N$, we can use the result from Corollary \ref{cor:3}, which under Assumption \ref{assum:4} can be expressed as:
\[
\alpha_k^* = 
\begin{cases}
	\frac{
		\left(\norm[\infty]{\gradApp{\vtheta}} - \nicefrac{d_{\delta}}{\sqrt{N}}\right)^2}
		{2c
		\left(\norm[\infty]{\gradApp{\vtheta}} + \nicefrac{d_{\delta}}{\sqrt{N}}\right)^2} & 
		\text{if } k = \arg\max_i |\gradApp{\theta_i}|,	\\
		0 & \text{otherwise},
\end{cases}
\]
which yields:
\[
	\Upsilon_{\delta}(\Lambda^*,N) = 
	\frac{
		\left(\norm[\infty]{\gradApp{\vtheta}} - \nicefrac{d_{\delta}}{\sqrt{N}}\right)^4}
		{4c
		\left(\norm[\infty]{\gradApp{\vtheta}} + \nicefrac{d_{\delta}}{\sqrt{N}}\right)^2N}.
\]
To justify the use of Corollary \ref{cor:3}, our $N^*$ must be compliant with \ref{assum:3}, which, under Assumption \ref{assum:4}, translates into the following constraint:
\[
N \geq N_0 := \frac{d_{\delta}^2}{\norm[\infty]{\gradApp{\vtheta}}^2}.
\] 
By computing the derivative $\nicefrac{\partial\Upsilon_{\delta}}{\partial N}$ we find just two stationary points in $[N_0,+\infty)$: the first one is $N_0$ itself, which is a minimum, since $\Upsilon_{\delta}(\Lambda^*,N_0) = 0$ and $\Upsilon_{\delta}$ is non-negative. The other stationary point is our optimal batch size:
\[
	N^* =  \frac{(13+3\sqrt{17})d_{\delta}^2}{2\norm[\infty]{\gradApp{\vtheta}}^2}.
\]
Since $\Upsilon_{\delta}(\Lambda^*,N_1)=0, \lim\limits_{N \to +\infty}\Upsilon_{\delta} (\Lambda^*,N)= 0$, and $\Upsilon_{\delta}$ is continuous and differentiable in $[N_0,+\infty)$, $N^*$ is indeed the global maximum in the region of interest. We can now substitute $N^*$ into $\Lambda^*$ to obtain:
\begin{align*}
&\alpha_k^* = 
\begin{cases}  
	\frac{(13-3\sqrt{17})}
		{4c} & 
		\text{if } k = \arg\max\limits_i |\gradApp{\theta_i}|,	\\
		0 & \text{otherwise},
\end{cases}
\end{align*}
and into $\Upsilon_{\delta}(\Lambda^*,N^*)$ to obtain:
\[
	\Upsilon^* = \frac{(4977-1207\sqrt{17})\norm[\infty]{\gradApp{\vtheta}}^4}{32d_{\delta}^2}.
\]
Finally
\[
	\DeltaJ \geq N^*\Upsilon^* 
	= \frac{393-95\sqrt{17}}{8}\norm[\infty]{\gradApp{\vtheta}}^2,
\]

with probability at least $(1-\delta)^m$.

\end{proof}


% \paragraph{Lemma \ref{lemma:2}}
\secondlemma*
\begin{proof}
We focus on the REINFORCE\cite{Williams1992} gradient estimator:
\[
\hat{\nabla}_{\vtheta}J_{\mu}^{RF}(\vtheta) = 
	\frac{1}{N}\sum\limits_{n=1}^N\left(
	\sum\limits_{k=1}^H\nabla_{\vtheta}\log\pi(a_k^n|s_k^n,\vtheta)
	\left(\sum\limits_{l=1}^H\gamma^{l-1}r_l^n-b\right)\right),
\]
and the G(PO)MDP\cite{bb-ihgbps-01}/PGT\cite{Sutton1999a} gradient estimator:
\[
\gradPGT{\vtheta} = \gradGPOMDP{\vtheta} = 
	\frac{1}{N}\sum\limits_{n=1}^N\left(
	\sum\limits_{k=1}^H\nabla_{\vtheta}\log\pi(a_k^n|s_k^n,\vtheta)
	\left(\sum\limits_{l=k}^H\gamma^{l-1}r_l^n-b_l^n\right)\right).
\]
In both cases, the i-th component of the single sample can be bounded in absolute value as
\[
\sum\limits_{k=1}^H\nabla_{\theta_i}|\log\pi(a_k^n|s_k^n,\vtheta)|\frac{R}{1-\gamma}.
\]
In the case of Gaussian policy, bounded action space ($|a| \leq \overline{A}$) and under Assumption \ref{assum:1}, in the most general case, the range of the term $\nabla_{\theta_i}\log\pi(a_k^n|s_k^n,\vtheta)$ can be bounded as
\[
\frac{2M_{\phi}\overline{A}}{\sigma^2}.
\] 
Finally
\[
\mathbf{R} \leq \frac{2HM_{\phi}\overline{A}R}{\sigma^2(1-\gamma)}.
\]
\end{proof}